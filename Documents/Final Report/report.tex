%%
%% This is file `sample-sigconf-biblatex.tex',
%% generated with the docstrip utility.
%%
%% The original source files were:
%%
%% samples.dtx  (with options: `all,proceedings,sigconf-biblatex')
%%
%% IMPORTANT NOTICE:
%%
%% For the copyright see the source file.
%%
%% Any modified versions of this file must be renamed
%% with new filenames distinct from sample-sigconf-biblatex.tex.
%%
%% For distribution of the original source see the terms
%% for copying and modification in the file samples.dtx.
%%
%% This generated file may be distributed as long as the
%% original source files, as listed above, are part of the
%% same distribution. (The sources need not necessarily be
%% in the same archive or directory.)
%%
%%
%% Commands for TeXCount
%TC:macro \cite [option:text,text]
%TC:macro \citep [option:text,text]
%TC:macro \citet [option:text,text]
%TC:envir table 0 1
%TC:envir table* 0 1
%TC:envir tabular [ignore] word
%TC:envir displaymath 0 word
%TC:envir math 0 word
%TC:envir comment 0 0
%%
%% The first command in your LaTeX source must be the \documentclass
%% command.
%%
%% For submission and review of your manuscript please change the
%% command to \documentclass[manuscript, screen, review]{acmart}.
%%
%% When submitting camera ready or to TAPS, please change the command
%% to \documentclass[sigconf]{acmart} or whichever template is required
%% for your publication.
%%
%%
\documentclass[sigconf,natbib=false]{acmart}
%%
%% \BibTeX command to typeset BibTeX logo in the docs
\AtBeginDocument{%
  \providecommand\BibTeX{{%
    Bib\TeX}}}

%% Rights management information.  This information is sent to you
%% when you complete the rights form.  These commands have SAMPLE
%% values in them; it is your responsibility as an author to replace
%% the commands and values with those provided to you when you
%% complete the rights form.
% \setcopyright{acmlicensed}
% \copyrightyear{2018}
% \acmYear{2018}
% \acmDOI{XXXXXXX.XXXXXXX}
%% These commands are for a PROCEEDINGS abstract or paper.
% \acmConference[Conference acronym 'XX]{Make sure to enter the correct
%   conference title from your rights confirmation email}{June 03--05,
%   2018}{Woodstock, NY}
%%
%%  Uncomment \acmBooktitle if the title of the proceedings is different
%%  from ``Proceedings of ...''!
%%
%%\acmBooktitle{Woodstock '18: ACM Symposium on Neural Gaze Detection,
%%  June 03--05, 2018, Woodstock, NY}
% \acmISBN{978-1-4503-XXXX-X/2018/06}


%%
%% Submission ID.
%% Use this when submitting an article to a sponsored event. You'll
%% receive a unique submission ID from the organizers
%% of the event, and this ID should be used as the parameter to this command.
%%\acmSubmissionID{123-A56-BU3}

%%
%% For managing citations, it is recommended to use bibliography
%% files in BibTeX format.
%%
%% You can then either use BibTeX with the ACM-Reference-Format style,
%% or BibLaTeX with the acmnumeric or acmauthoryear sytles, that include
%% support for advanced citation of software artefact from the
%% biblatex-software package, also separately available on CTAN.
%%
%% Look at the sample-*-biblatex.tex files for templates showcasing
%% the biblatex styles.
%%


%%
%% The majority of ACM publications use numbered citations and
%% references, obtained by selecting the acmnumeric BibLaTeX style.
%% The acmauthoryear BibLaTeX style switches to the "author year" style.
%%
%% If you are preparing content for an event
%% sponsored by ACM SIGGRAPH, you must use the acmauthoryear style of
%% citations and references.
%%
%% Bibliography style
\RequirePackage[
  datamodel=acmdatamodel,
  style=acmnumeric,
  ]{biblatex}

\addbibresource{references.bib}

\usepackage{graphicx}
\graphicspath{{./images/}}

\setcopyright{none}
\copyrightyear{2026}
\acmYear{2026}
\acmDOI{}
\acmConference[Course Project Report]{CHI-Style Course Project Report}{January 2026}{Ljubljana, Slovenia}
\acmISBN{}

%%
%% end of the preamble, start of the body of the document source.
\begin{document}

%%
%% The "title" command has an optional parameter,
%% allowing the author to define a "short title" to be used in page headers.
\title{Improving the Usability of Self-Checkout Systems
	in Retail Stores}

\author{Stefan Krstevski}
\affiliation{%
	\institution{University of Ljubljana}
    \city{Ljubljana}
    \country{Slovenia}
}
\author{Primož Mihelak}
\affiliation{%
	\institution{University of Ljubljana}
    \city{Ljubljana}
    \country{Slovenia}
}
\author{Jan Rojc}
\affiliation{%
	\institution{University of Ljubljana}
    \city{Ljubljana}
    \country{Slovenia}
}
\author{Tjaž Štok}
\affiliation{%
	\institution{University of Ljubljana}
    \city{Ljubljana}
    \country{Slovenia}
}

%%
%% By default, the full list of authors will be used in the page
%% headers. Often, this list is too long, and will overlap
%% other information printed in the page headers. This command allows
%% the author to define a more concise list
%% of authors' names for this purpose.
% \renewcommand{\shortauthors}{Trovato et al.}

\begin{abstract}
Self-checkout systems are widely used in retail environments, yet users often
experience uncertainty, errors, and accessibility issues due to unclear
instructions, poor visibility of key information, and limited user control.
This project investigates the user experience of self-checkout machines in
Slovenian grocery stores. We analyzed existing systems used in Mercator, Spar,
and dm through field observations and an initial online survey with 21
participants from diverse age groups.

Based on the identified usability issues, we designed and iteratively refined a
self-checkout interface prototype. The design process was informed by feedback
from usability testing, classmate and assistant reviews, and a second survey conducted
after users tested the final prototype (11 participants). The results revealed
recurring problems in existing systems, including unclear icons, low
visibility of important information, small touch targets, and insufficient
support for error recovery.

Our final design proposal addresses these issues through an improved interaction
flow and an interface focused on visibility of system status, recoverability, and
accessibility. This report summarizes the research and design process, while the
accompanying portfolio \cite{github-portfolio} contains collected data,
prototypes, and additional materials.
\end{abstract}

% CCS / Keywords
\begin{CCSXML}
<ccs2012>
 <concept>
  <concept_id>10003120.10003121.10003122</concept_id>
  <concept_desc>Human-centered computing~HCI design and evaluation methods</concept_desc>
  <concept_significance>500</concept_significance>
 </concept>
 <concept>
  <concept_id>10003120.10003121.10003121.10003125</concept_id>
  <concept_desc>Human-centered computing~Usability testing</concept_desc>
  <concept_significance>300</concept_significance>
 </concept>
</ccs2012>
\end{CCSXML}

\ccsdesc[500]{Human-centered computing~HCI design and evaluation methods}
\ccsdesc[300]{Human-centered computing~Usability testing}

\keywords{self-checkout, usability, user experience, retail HCI}

\maketitle

\section{Introduction}
Self-checkout machines allow customers to scan and pay for products without direct
cashier assistance. While such systems aim to improve efficiency and reduce queues,
they can be difficult to use due to unclear instructions, insufficient feedback and
limited possibilities for recovering from mistakes. These issues are especially
problematic for first-time users and older customers, who often report lower
confidence and higher error rates when using self-checkout systems.

The goal of this project was to design an improved self-checkout interface that
reduces user errors and improves clarity, control, and overall usability.

This report provides a concise overview of the research and design process. All
supporting materials, including raw data, intermediate results, and design
artifacts, are documented in the accompanying portfolio.

\section{Course Context}
This work was conducted as part of the course \emph{Komunikacija človek--računalnik}.
The project was completed during the semester following the prescribed schedule,
including prototyping activities in Week 8, usability testing in Week 9, and an
intermediate presentation of progress in Week 10. The final report and portfolio
together constitute the required deliverables for the course project.

\section{Methodology}
To design an improved prototype, we first observed the existing self-checkout solutions in Mercator, Spar (Figure~\ref{fig:spar-checkout}) and dm (Figure~\ref{fig:dm-checkout}). We took photos of all of the screens that these systems use so that we could analyze them later in more detail. After identifying the problems and also the positives, we created an early brainstormed sketch of our proposed solution. We then conducted a survey, which had 21 participants, about the problems people encountered when using these systems. When we had our first prototype in Figma ready, we conducted usability testing to gather feedback and find the issues we may have overlooked. The prototype was designed iteratively, we improved on each iteration by receiving feedback and fixing the issues. When our prototype was finished, we asked the users to test it and then conducted another survey about our solution. The results would let us know if we were successful at making an improvement of the existing solutions.

%We followed a user-centered design process consisting of the following stages:
%\begin{enumerate}
%    \item field observation of existing self-checkout systems,
%   \item user research through an online survey,
%   \item iterative prototyping, and
%   \item informal usability testing.
%\end{enumerate}

\subsection{Observations}

While several usability problems were observed across systems, the self-checkout
interface used in dm stores stood out as particularly clear and easy to follow.
Compared to the other systems, the dm interface provided a more consistent visual
hierarchy, larger and clearer action buttons, and better step-by-step guidance. Its interface was also less cluttered, which made it more visually appealing.
It served as a positive reference and inspiration during the design phase.

\begin{figure}[ht]
    \centering
    \includegraphics[width=\linewidth]{spar6.jpg}
    \caption{Self-checkout interface observed at Spar, highlighting unclear
    icons and low visibility of important information.}
    \label{fig:spar-checkout}
\end{figure}

\begin{figure}[ht]
    \centering
    \includegraphics[width=\linewidth]{dm3.jpeg}
    \caption{Self-checkout interface observed at dm. The clear visual hierarchy
    and step guidance served as inspiration for our redesign.}
    \label{fig:dm-checkout}
\end{figure}

\subsection{Survey Design and Participants}
To complement the observations, we conducted two online surveys using Google Forms.
The first survey focused on users’ experiences with existing self-checkout systems
and collected demographic data (age group), frequency of self-checkout usage,
perceived clarity of instructions and interaction flow, and open-ended feedback on
what users liked and found frustrating. The first survey received 21 responses.

After completing the final design prototype, we conducted a second survey aimed at
evaluating the redesigned interface. Participants were asked to use the prototype
and then answer questions related to visual clarity, interaction flow, error
recovery, and overall confidence while completing checkout tasks. This second
survey received 11 responses.

Both surveys included participants from multiple age groups. Some responses were
submitted in Slovenian and later translated during analysis. The full surveys
and raw response datasets for both surveys are included in the portfolio.

\subsection{Analysis Approach}
Responses from both surveys were analyzed using the autogenerated charts provided by
Google Forms. Open-ended answers were manually grouped into themes such as
navigation, feedback, readability, accessibility, and error recovery. This
lightweight analysis approach was sufficient to identify patterns and to
compare issues in existing systems with the performance of our redesigned
prototype.

\section{Results}
Results from the first survey and field observations revealed several consistent
usability problems in existing self-checkout systems. Users frequently reported
that important information, such as the total price or the current step in the
process, was not visible enough. Icons and labels were often described as unclear,
and several respondents noted that small buttons and low color contrast made
interaction difficult, particularly for older users. An excerpt of the first survey
results is shown in Figure~\ref{fig:survey1-chart}.

\begin{figure}[ht]
    \centering
    \includegraphics[width=\linewidth]{survey1-results.png}
    \caption{Answers to survey questions regarding usability issues in existing
    self-checkout systems.}
    \label{fig:survey1-chart}
\end{figure}

In addition to visual issues, many users experienced problems recovering from
mistakes. Survey responses and observations highlighted the lack of a clear back or
undo option, difficulty removing items from the cart, and confusion during payment
or when starting the session.

Results from the second survey, conducted after survey participants tested our final design
prototype, indicate a clear improvement in usability. When asked about visual
issues, most participants reported no major problems, with only a small number
still noting unclear icons or text. Notably, when asked whether they were unsure
what to do next at any step while using the prototype, all 11 respondents answered
that they did not experience any uncertainty. The results of these two key
questions from the second survey are shown in Figure~\ref{fig:survey2-chart}.

\begin{figure}[ht]
    \centering
    \includegraphics[width=\linewidth]{survey2-results.png}
    \caption{Results from the second survey after testing the final prototype,
    showing reduced visual issues and improved clarity of interaction flow.}
    \label{fig:survey2-chart}
\end{figure}

These findings suggest that the design changes introduced in the final prototype
successfully addressed many of the usability issues identified in the first survey,
particularly those related to visibility, clarity of next steps, and user
confidence.

\section{Iterative Design Process}
Our design process progressed through three clearly defined stages: early sketches,
a usable prototype, and a final design proposal. At each stage, we incorporated
feedback from presentations, usability testing, our classmates, and the assistant.

\subsection{Early Sketches}
We began with early sketches to brainstorm alternative
layouts and navigation routes. These sketches explored ideas such
as clearer bag selection, visible language and sound controls, and improved back
actions. Feedback from the assistant at this stage helped us identify
the disadvantages of the bag selection screen, redundancy between actions
(e.g., back vs.\ discard session), and the need for improved accessibility features.
An example of an early sketch is shown in Figure~\ref{fig:first-sketch}.

\begin{figure}[ht]
    \centering
    \includegraphics[width=\linewidth]{first-sketch.png}
    \caption{Early brainstormed sketch exploring layout and interaction ideas.}
    \label{fig:first-sketch}
\end{figure}

\subsection{Usable Prototype}
Based on the early sketches and the assistant's feedback, we developed a usable interactive
prototype in Figma. This prototype implemented a complete checkout flow and allowed
users to scan items, select bags, and proceed to payment. During the mid-semester
presentation and usability testing we received extensive feedback from our classmates and assistant that
helped us with several iterations of the prototype.

Main improvements were a simplified bag selection screen (defaulting to zero
bags, adding clear +/- controls, and placing the confirmation button consistently),
making language and sound controls visible on all screens, and adding labels to icons
(e.g. the speaker icon). We also removed unnecessary elements, such as the item list on the
payment screen, and reconsidered the need for a dedicated “start scanning” button,
giving users a clear hint how to start the self-checkout process.

Further iterations addressed feedback on interactions and user control.
For example, users are prompted to confirm item removal and select bag, and personalization options such as disabling receipts were
considered. Based on the assistant's feedback, functionally different buttons were no
longer grouped together, icons were added to the bottom row buttons (discard, continue,
call assistant), and scanning was bound to the space bar instead of a large
on-screen button. The resulting usable prototype is shown in
Figure~\ref{fig:usable-prototype}.

\begin{figure}[ht]
    \centering
    \includegraphics[width=\linewidth]{early-prototype.png}
    \caption{Usable interactive prototype refined through presentation feedback,
    usability testing, and the assistant's guidance.}
    \label{fig:usable-prototype}
\end{figure}

\subsection{Final Design}
The final design represents the outcome of iterative feedback from
surveyed users, our classmates, the assistant, and professional guidance, resulting
in a design that emphasizes visibility of system status, user control, and
accessibility. Building on earlier iterations, key design decisions include a
persistent back button, a clear step indicator showing progress through the checkout
process, and simple controls for removing items from the cart.

Visual hierarchy was simplified using larger text, clearer spacing, improved
contrast, and more consistent icons to support a wider range of users,
including those with accessibility needs. For example, the interface includes
larger touch targets, and labeled icons.

An overview of the redesigned interface is shown in Figure~\ref{fig:final-prototype}.

The complete prototypes are included in the portfolio.

\begin{figure}[ht]
    \centering
    \includegraphics[width=\linewidth]{final-prototype.png}
    \caption{Final design proposal developed with professional UI/UX guidance.}
    \label{fig:final-prototype}
\end{figure}

\section{Evaluation}
We conducted informal usability testing with participants performing typical
checkout tasks using the final prototype. Testing focused on task completion,
error recovery, and user confidence. Compared to existing systems, participants
found it easier to understand what to do next and how to recover from mistakes.
Testing notes and participant feedback are documented in the portfolio.

% Beyond task completion and error recovery, we also noted qualitative improvements
% in user confidence and perceived efficiency. Participants reported feeling more in
% control and less anxious about making mistakes. While formal metrics were not
% collected due to course constraints, these subjective impressions support the
% effectiveness of the iterative, user-informed design approach.

\section{Discussion}
Our findings suggest that the most significant usability issues in self-checkout
systems are not caused by technical complexity, but by insufficient feedback and
limited user control. Problems related to undoing actions, understanding system
state, and navigating between steps were more important than scanning or payment
mechanics themselves. These observations are reflected in our second survey results,
where no participants reported uncertainty about the next step.

An interesting observation was that users tended to rely heavily on visual cues
rather than text instructions, highlighting the importance of intuitive iconography
and consistent step indicators. This insight reinforced our decision to prioritize
visual hierarchy and clear status feedback in the final design.

\section{Limitations}
This project was conducted as part of a semester-long course and therefore had a
limited sample size and informal testing setup. Differences between store
implementations and hardware constraints may also influence user experience.
Future work should include larger-scale testing and formal accessibility
evaluation.

\section{Conclusion}
This project demonstrates that relatively simple, research-driven interface
changes can significantly improve the usability of self-checkout systems. Through
field observations, a user survey, and iterative prototyping, we developed a
redesign that improves clarity, control, and accessibility. The report summarizes
the process and key results, while the accompanying portfolio provides full
documentation and supporting materials.

\begin{acks}
This work was completed as a course project. We thank all survey participants,
testers, and the assistant for their feedback and guidance.
\end{acks}

\printbibliography

%%
%% If your work has an appendix, this is the place to put it.
% \appendix

\end{document}
\endinput
